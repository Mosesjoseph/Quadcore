\documentclass[a4paper,12pt]{article}
\usepackage{blindtext}
\usepackage[utf8]{inputenc}
\usepackage{graphicx}
\usepackage{enumitem}
\usepackage{booktabs}
\usepackage{verbatim}
\usepackage{makecell}

\begin{document}
\begin{titlepage}
\center

\textsc{\LARGE Application Design and Functional Requirements}\\[1.5cm]
\textsc{\Large Project: Traffic Camera Image Analysis}\\[1.5cm]
\textsc{\large Client: DPSS, CSIR}\\[0.5cm]
\textsc{\large Team: Quadcore Productions}\\[0.5cm]

\begin{minipage}{0.4\textwidth}
\begin{flushleft} \large
\textbf{Author(s):}\\
Mpho \textsc{Baloyi}\\
Hlengekile \textsc{Jita}\\
Mayimela \textsc{Moses}\\
Mbhele \textsc{Themba}\\
\end{flushleft}
\end{minipage}
~
\begin{minipage}{0.4\textwidth}
\begin{flushright} \large
\textbf{Student number(s):} \\
14133670\\ % Student number
14077893\\
14019702\\
14007950\\
\end{flushright}
\end{minipage}\\

\includegraphics[width=\textwidth]{2012-quad-core-phones.jpg}

{\large University of Pretoria, Department of Computer Science}\\

{\large 22 October 2016}\\[3cm]

\vfil

\end{titlepage}

\newpage
\tableofcontents
\newpage

\newpage
%\begin{comment}
\begin{table}[ht]
 \centering
 \caption{Version History}
 \label{tab:table1}
 \begin{tabular}{cccc}
   \toprule Version No. & Date & Changes & By\\
   \midrule 1 & 27 May 2016 & \makecell {Introduction,\\ Background,\\ Vision,\\ Scope,\\ Use cases} & \makecell {Mpho Baloyi,\\ Hlengekile Jita,\\ Moses Mayimela,\\ Themba Mbhele}\\
   2 & 29 July 2016 & \makecell {Updated Scope,\\ Updated Use Cases,\\ Added Service Contracts,\\ Added Required Functionality,\\ Added Process Specification} & \makecell {Mpho Baloyi,\\ Hlengekile Jita,\\ Moses Mayimela,\\ Themba Mbhele} \\
   3 & 22 October 2016 & \makecell {Added Extra Route Info Use Case,\\ Added Captions and Figure References,\\ Changed Open Issues} & \makecell {Mpho Baloyi,\\ Hlengekile Jita,\\ Moses Mayimela,\\ Themba Mbhele} \\
   \bottomrule
  \end{tabular}
\end{table}
%\end{comment}
\newpage

\section{Introduction}
This document describes the functional requirements and application design of the Traffic Camera Image Analysis System. The target system will run on a web server  and be accessed by users on an Android Application which will provide the users with the necessary functionality to access real-time traffic information that assists them with things such as avoiding traffic and choosing the best alternative routes. 

In this specification, we cover the use cases, their service contracts, required functionality and process specifications of the target system. The above, will be built on over time as the software is developed, as we are following an agile development methodology. In addition the domain model of the application as a whole will be provided.
\section{Vision}
For this project we aim to achieve a system that makes use of images obtained from highway cameras to provide users with up-to-date real-time traffic information. The system should simplify the user's travels by providing traffic information and notifying them before they depart of traffic conditions, calculating arrival times based on traffic conditions and help them select the most suitable route for their trips using the traffic information and additional metrics. Our vision for the target system is that it should be reliable and perform relatively quickly, both for user satisfaction and in order to be the an up-to-date traffic information system.
\section{Background}
As a commuter, traffic is something that is a part of everyday life, and it is not one of the more pleasurable aspects of life. Already there is software in place that assists us in dealing with this problem, such as Google Maps. This software uses the crowd-sourcing of GPS data in order to provide their up-to-date traffic information.

In our alternative system, we take an image analysis approach to solve the same problem. We will make use of the publicly available SANRAL highway cameras to get images. Processing these images we will perform image analysis and determine the traffic conditions in the area of a camera. Using this information we will be able to generate information pertaining to user specified routes in order to provide the information necessary to help them avoid traffic and choose the most suitable routes in order to do so.
\section{System Scope}
The purpose of this system is to provide the user with traffic information based on image analysis from images accessed from highway cameras. The system will need to provide the user with services that allow the achievement of this goal.

The scope of the system which is depicted below in Figure ~\ref{fig:SystemScope} includes the tasks the user can access such as add route, modify route, delete route and get route traffic. This document will further describe each of these tasks as system use cases in the functional requirements section.
\begin{figure}[here]
\includegraphics[width=\textwidth]{images/System_Scope.jpg}
\caption{System Scope}
\label{fig:SystemScope}
\end{figure}

\section{Functional Requirements}
\subsection{Use Cases}
The use cases for our system are described below. These use cases are the services that provide the user of the application some value. The following uses are listed in order of prioritization:
\begin{itemize}
\item Add Route 			(Critical)
\item Get Route Traffic		(Critical)
\item Modify Route			(Critical)
\item Delete Route			(Critical)
\item Schedule Notification (Important)
\item Extra Route Information 	(Nice To Have)
\end{itemize}
\subsection{Service Contracts}
For each of the above services described above, there are pre-conditions that must be met for the service to be carried out and post-conditions that are met to indicate the successful completion of a service. In addition the service contracts indicate what exceptions will be thrown in the case that a service fails.

\subsubsection{Add Route}
\begin{figure}[here]
\includegraphics[width=\textwidth]{images/scAdd_Route.jpg}
\caption{Service Contract for Add Route Service}
\label{fig:sc_addroute}
\end{figure}

\textbf{Pre-Condition: }
Valid route boundaries.\\
Route does not already exist.\\
\textbf{Post-Condition: }
New route is created and stored.\\

The Add Route service contract depicted above in Figure ~\ref{fig:sc_addroute} shows the use case's pre and post conditions. This use case creates a new route and stores it so that it can be used in requests for traffic information. As a pre-condition it first checks that the route to be created is valid and does not already exist. Thereafter it creates and stores the route, meeting the post-condition. In the case that the route exists an exception is raised.
\subsubsection{Modify Route} 
\begin{figure}[here]
\includegraphics[width=\textwidth]{images/scModify_Route.jpg}
\caption{Service Contract for Modify Route Service}
\label{fig:sc_modifyroute}
\end{figure}

\textbf{Pre-Condition: }
Route exists.\\
\textbf{Post-Condition: }
Route is modified and the modified route is stored.\\

This use case makes changes to existing routes and stores the changes. The service contract depicted above in Figure ~\ref{fig:sc_modifyroute} shows the use case's pre and post conditions. This use case first checks if the route exists, if so it then makes the necessary changes to the route and then stores the route. If the route does not exist an exception is raised. 
\subsubsection{Delete Route}

\begin{figure}[here]
\includegraphics[width=\textwidth]{images/scDelete_Route.jpg}
\caption{Service Contract for Delete Route Service}
\label{fig:sc_deleteroute}
\end{figure}

\textbf{Pre-Condition: }
Route exists.\\
\textbf{Post-Condition: }
Route is removed\\
This use case, whose service contract is depicted above in Figure ~\ref{fig:sc_deleteroute}, removes an existing route. Before it does this it checks if the route exists in storage then removes it. In the case that the route does not exist an exception is raised.
\subsubsection{Get Route Traffic}
\includegraphics[width=\textwidth]{images/scDetermine_Traffic.jpg}
\begin{figure}[here]
\includegraphics[width=\textwidth]{images/scDelete_Route.jpg}
\caption{Service Contract for Get Route Traffic Service}
\label{fig:sc_getroutetraffic}
\end{figure}

\textbf{Pre-Condition: }
Route consists of camera locations.\\
Camera locations on route has valid images.\\
\textbf{Post-Condition: }
Traffic information for route is generated.\\

The Get Route Traffic service contract depicted above in Figure ~\ref{fig:sc_getroutetraffic} shows the use case's pre and post conditions. This use case generates traffic information based on a route provided with the request. To do this it first checks if the route covers any of the camera locations, if not no traffic information is generated and an exception is raised. Otherwise it continues and retrieves the images for the camera's within the route, thereafter it checks if these images are valid, if all are invalid, no traffic information is generated, it then performs image analysis on the valid images and determine the traffic levels and generates traffic information for the route.

\subsubsection{Schedule Notification}
\includegraphics[width=\textwidth]{images/scSchedule_Notification.jpg}
\begin{figure}[here]
\includegraphics[width=\textwidth]{images/scDelete_Route.jpg}
\caption{Service Contract for Schedule Notification Service}
\label{fig:sc_notification}
\end{figure}

\textbf{Pre-Condition: }
Valid times\\
\textbf{Post-Condition: }
Notification is scheduled\\

This use case allows the user to schedule notifications of traffic information at specific times. It's service contract is depicted in Figure ~\ref{fig:sc_notification}. This use case is completed by allowing the user to specify a time to be notified about a route when adding or modifying it. This time is then used to schedule automated traffic information requests.

\subsubsection{Extra Route Information}
\includegraphics[width=\textwidth]{images/scExtra_Route_Info.jpg}

\begin{figure}[here]
\includegraphics[width=\textwidth]{images/scDelete_Route.jpg}
\caption{Service Contract for Extra Route Information Service}
\label{fig:sc_extra}
\end{figure}

\textbf{Pre-Condition: }
Get Route Traffic use case succeeds.\\
Route has roadworks and incident information.\\
\textbf{Post-Condition: }
Extra traffic information for route is generated.\\

The Extra Route Information service contract depicted in Figure ~\ref{fig:sc_extra} shows what pre- and post- conditions must be met to recieve extra route information, specifically information about roadworks and incidents. This service is completed if and only if the Get Route Traffic service is completed as an addittional add on. Before the service can run, the system also has to check if such information for that route is available. If the service is able to run successfuly, the post-condition is met.

\subsection{Required Functionality}
The use cases make use of other lower level functions in order to achieve their purpose. In the below diagrams for each high level use case, the lower level use cases that are required are described.

\subsubsection{Add Route}
\begin{figure}[here]
\includegraphics[width=\textwidth]{images/Add_Route.jpg}
\caption{Use Case Diagram for the Add Route Service}
\label{fig:uc_add}
\end{figure}

The Add Route use case makes use of the following lower level functions depicted in Figure ~\ref{fig:uc_add} above to add a route to the system:
\begin{itemize}
\item \textbf{getRoute -} This checks if the route exists already or not.
\item \textbf{createRoute -} This creates a route, or a set of directions, based on the route boundaries (departure and destination points).
\item \textbf{persistRoute -} This stores the route so that it is available for later use.
\end{itemize}

\subsubsection{Modify Route} 
\begin{figure}[here]
\includegraphics[width=\textwidth]{images/Modify_Route.jpg}
\caption{Use Case Diagram for the Modify Route Service}
\label{fig:uc_modify}
\end{figure}

The Modify Route use case makes use of the following lower level functions depicted in Figure ~\ref{fig:uc_modify} above to make changes to an already existing route.
\begin{itemize}
\item \textbf{getRoute -} This checks if the route exists already.
\item \textbf{createRoute -} This creates a route, or a set of directions, based on the route boundaries (departure and destination points). Only if the changes are in the route boundaries.
\item \textbf{persistRoute -} This stores the modified route.
\end{itemize}

\subsubsection{Delete Route}

\begin{figure}[here]
\includegraphics[width=\textwidth]{images/Delete_Route.jpg}
\caption{Use Case Diagram for the Delete Route Service}
\label{fig:uc_delete}
\end{figure}

The Delete Route use case makes use of the following lower level functions depicted in Figure ~\ref{fig:uc_delete} above to assist in removing the route from storage:
\begin{itemize}
\item \textbf{getRoute -} This checks if the route exists already .
\end{itemize}

\subsubsection{Get Route Traffic}
\begin{figure}[here]
\includegraphics[width=\textwidth]{images/Get_Route_Traffic.jpg}
\caption{Use Case Diagram for the Get Route Traffic Service}
\label{fig:uc_gettraffic}
\end{figure}

The Get Route Traffic use case makes use of the following lower level functions, depicted in Figure ~\ref{fig:uc_gettraffic} above, both within and outside the Traffic Information module to determine the traffic of a route and send it back to the user:
\begin{itemize}
\item \textbf{checkForCameras -} This function checks if the route passes b y any of the available cameras.
\item \textbf{fetchImage -} This function fetches an image from the camera at the specified location.
\item \textbf{analyseImage -} This function performs image analysis in order to determine the traffic level in that location
\item \textbf{composeTrafficInfo -} This function composes all the traffic levels that make up the route and constructs a preliminary "traffic info report".
\item \textbf{sendTrafficInformation -} This function creates a traffic information report and sends it to the user who requested it.
\item \textbf{storeTrafficLevels -} The traffic levels are stored for any other user to be able to access it if their request requires similar information.
\end{itemize}

\subsubsection{Extra Route Information}
\begin{figure}[here]
\includegraphics[width=\textwidth]{images/Extra_Route_Info.jpg}
\caption{Use Case Diagram for the Extra Route Information Service}
\label{fig:uc_extra}
\end{figure}

The Extra Route Information use case makes use of the following lower level functions, depicted in Figure ~\ref{fig:uc_extra} above, to get roadworks and incident information for the route:

\begin{itemize}
\item \textbf{fetchRoadworksInfo -} This function fetches the roadworks information, if any, of the specified route.
\item \textbf{fetchIncidentInfo -} This function fetches he incident information, if any, of the specified route.
\item \textbf{sendTrafficInformation -} This function sends the traffic information to the user.
\item \textbf{storeTrafficInfo -} The traffic information is stored in cache for any other user who makes a similar request.
\end{itemize}

\subsection{Process Specification}
This section describes the process for completing the Add Route, Modify Route and Get Route Traffic Sections.
\subsubsection{Add Route}
The process for adding a route is shown below in Figure ~\ref{fig:ps_add}.
\begin{figure}[here]
\includegraphics[width=\textwidth]{images/psAdd_Route.jpg}
\caption{Process Specification Diagram for the Add Route Service}
\label{fig:ps_add}
\end{figure}
\subsubsection{Modify Route}
The process for modifying a route is shown below in Figure ~\ref{fig:ps_modify}.
\begin{figure}[here]
\includegraphics[width=\textwidth]{images/psModify_Route.jpg} 
\caption{Process Specification Diagram for the Modify Route Service}
\label{fig:ps_modify}
\end{figure}
\subsubsection{Get Route Traffic}
The process for getting traffic information is shown below in Figure ~\ref{fig:ps_gettraffic}.
\begin{figure}[here]
\includegraphics[width=\textwidth]{images/ps_GetRouteTraffic.jpg}
\caption{Process Specification Diagram for the Get Route Traffic Service}
\label{fig:ps_gettraffic}
\end{figure}

\section{Open Issues}
Possible future work to be done on the system:
\begin{itemize}
\item \textbf{Camera's from more sources} - The system only has access to SANRAL highway cameras. This limits the scope of the application. If other sources of camera's are found further work could include improving the analysis by having more images.
\end{itemize}
\end{document}

