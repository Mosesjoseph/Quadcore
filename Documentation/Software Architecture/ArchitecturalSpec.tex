\documentclass[a4paper,12pt]{article}
\usepackage{blindtext}
\usepackage[utf8]{inputenc}
\usepackage{graphicx}
\usepackage{enumitem}

\begin{document}
\begin{titlepage}
\center

\textsc{\LARGE Architectural Requirements}\\[1.5cm]
\textsc{\Large Project: Traffic Camera Image Analysis}\\[1.5cm]
\textsc{\large Client: DPSS, CSIR}\\[0.5cm]
\textsc{\large Team: Quadcore Productions}\\[0.5cm]

\begin{minipage}{0.4\textwidth}
\begin{flushleft} \large
\emph{Author(s):}\\
Mpho \textsc{Baloyi}\\
Hlengekile \textsc{Jita}\\
Mayimela \textsc{Moses}\\
Mbhele \textsc{Themba}\\
\end{flushleft}
\end{minipage}
~
\begin{minipage}{0.4\textwidth}
\begin{flushright} \large
\emph{Student number(s):} \\
14133670\\ % Student number
14077893\\
14019702\\
14007950\\
\end{flushright}
\end{minipage}\\

\includegraphics[width=\textwidth]{2012-quad-core-phones.jpg}

{\large University of Pretoria, Department of Computer Science}\\

{\large 27 May 2016}\\[3cm]

\vfil

\end{titlepage}

\newpage
\tableofcontents
\newpage
\section{Introduction}
This document describes the architectural requirements of the Traffic Camera Image Analysis System. The target system will run on a web server  and be accessed by users on an Android Application which will provide the users with the necessary functionality to access real-time traffic information that assists them with things such as avoiding traffic and choosing the best alternative routes.

In this specification we will cover the architectural scope of the system, the quality requirements, integration and access channel requirements and architectural constraints. In addition we will describe the software architecture we will use by describing the architectural patterns and tactics we will use as well as which frameworks and technologies will be used to achieve them.
\section{Vision}
For this project we aim to achieve a system that makes use of images obtained from highway cameras to provide users with up-to-date real-time traffic information. The system should simplify the user's travels by providing traffic information and notifying them before they depart of traffic conditions, and help them select the most suitable route for their trips using the traffic information and additional metrics. Our vision for the target system is that it should be reliable and perform relatively quickly, both for user satisfaction and in order to be the an up-to-date traffic information system.
\section{Background}
As a commuter, traffic is something that is a part of everyday life, and it is not one of the more pleasurable aspects of life. In our system,we will take an image analysis approach to attempt to solve this problem. We will make use of the publicly available SANRAL highway cameras to get images which can be found on https://www.i-traffic.co.za/traffic/cameras.aspx. We will perform image analysis on these images and determine the traffic conditions in the area of a camera. With this analysis, we will be able to generate information pertaining to user specified routes in order to help them avoid traffic.
\section{Architectural Requirements}
\subsection{Architectural Scope}
The system will consist of a back-end and front-end.
\subsubsection{Back-end}
\subsubsection{Front-end}
The front-end of the system, the android mobile application, will be used to obtain information from the user. The information that will gathered includes route information (start and end locations) and scheduling information (when a traffic report will be automatically pushed to the user). This obtained information will have to be persisted.\\
The information that is gathered from the user will be used to generate traffic information. This requires that a reporting mechanism be in place so that the traffic information can be relayed to the user. 
\subsection{Quality Requirements}
\subsubsection{Usability}
Usability is a key component of the system because user satisfaction is a high priority. The aim is to create an application that is easy to understand and thus, easy to learn how to us. \\
\textbf{How to achieve usability}
\begin{itemize}
\item The interface of the mobile application will be designed in an intuitive manner meaning that it will be easy to navigate and easy to understand.It will include buttons and other methods of interaction for which their functionality is self explanatory.
\item The interface will only have what is required and nothing more to ensure that it is not cluttered and does not cause confusion.\\
\end{itemize}
\subsubsection{Scalability}
Scalability refers to a systems ability to handle a heavy workload. This is an important aspect of the system because reports will have to be provided at very fast rates. Multiple users will be making use of the system at the same time and they all need to be catered to at the same levels of performance. \\
\textbf{How to achieve usability}
\begin{itemize}
\item To improve the performance of the system, reports will only be generated when it is necessary to do so. To elaborate on this, the website from which we will be getting the images from updates its images at one minute intervals so between that period, we will not attempt to grab new images but will make use of the images that were grabbed at the previous interval. In this way, no unnecessary processing takes place.
\item We will not generate a new report if it is not necessary to do so. If we generate a report for one user and another user requests a report for the exact same route when it is not time to update the images, we will use the same report that was generated for the first user. We will do this by implementing a cache server to store route reports.  
\end{itemize}
\subsubsection{Maintainability}
The components of the system need to implemented in a manner that makes it easy for them to be updated or replaced.\\
\textbf{How to achieve maintainability}
\begin{itemize}
\item Maintainability will be achieved by writing loosely coupled code.
\end{itemize}
\subsubsection{Reliability}
Reliability is of utmost importance because a lot of people will be using the application, especially at traffic peak hours thus the system has to get the job done and it should do it effectively so that users get the best service possible. \\
\textbf{How to achieve reliability} \\
\begin{itemize}
\item The main focus will be placed on the prevention of faults. This will be done by testing the system extensively and working towards fixing any fault that is encountered.
\end{itemize}
\subsection{Integration and Access Channel Requirements}
\subsubsection{Integration Requirements}
The system is integrated with some services that are provided by Google. The protocol that will be used to facilitate communication between our system and Google are https requests made to a server.\\
The system is also integrated with SANRAL. Communication between the two also occurs via https requests made to a server.\\
\subsubsection{Access Channel Requirements}
The system is available to clients in the form of a mobile application. The mobile application allows clients to set up routes for which they would like to receive traffic updates on. Once these routes have been set up, the client is able to switch between the routes and a traffic report will be generated on the selected route. The report that is generated is then conveyed to the client in both a graphical and a textual manner.

\subsection{Architectural Constraints}
The server provided by SANRAL, which we pull all of our traffic data from, is slow, and there are days when the server is unavailable at all.
\subsubsection{Technologies}
\begin{itemize}
\item Python: Python will be used to implement the web server of the system which will perform the necessary processing of data and after completion, the server will send the data to the mobile application via Google Cloud Messaging.
\item OpenCV: OpenCV is an image processing library that the server will make use of to process the traffic images from SANRAL.
\item Android Studio: Android Studio will be used to design a mobile application on which traffic reports will be displayed.
\item SQLite, a self-contained, serverless database that requires no configuration, will be used to store all user generated data locally on each device
\end{itemize}
\section{Initial Software Architecture Design}
\includegraphics[width=\textwidth]{Layers.jpg}
\end{document}
